\documentclass[letterpaper, 10 pt, conference]{ieeeconf}  % Comment this line out
\usepackage[utf8]{inputenc}
\usepackage[T1]{fontenc}  
% if you need a4paper
%\documentclass[a4paper, 10pt, conference]{ieeeconf}     
% Use this line for a4

\IEEEoverridecommandlockouts         % use the \thanks command
\overrideIEEEmargins

\title{\LARGE \bf Processador Pipeline}


\author{Acássio, André, Caetano, Erick, Everton, Felipe, Virgínia.}

\begin{document}

\maketitle 
\thispagestyle{empty}
\pagestyle{empty}

\begin{abstract}

\end{abstract}

\section{INTRODUÇÃO}

O projeto visa desenvolver noções básicas de operacionalização das partes de um processador de propósito geral, otimizado com pipeline de instruções.

\section{ESPECIFICAÇÃO DO PROCESSADOR}

\begin{itemize}
\item Arquitetura de 5 estágios funcionais de execução paralela.
\item Possibilidade de lidar com 3 hazards RAW consecutivos, sem paralisar a propagação de instruções entre os estágios funcionais.
\item Instruções de 32 bits e operações internas em 16 bits.
\item Mínimo de 32 registradores de propósito geral: R0 - R31(R0 e R1 auxiliares).
\item Memória segmentada, com acesso definido a cada segmento de acordo com a instrução utilizada:
\item Modos de endereçamento: 
\begin{itemize}
\item imediato
\item base-deslocamento
\item a registrador
\end{itemize}

\item Comunicação via mecanismo de interrupção:
\begin{itemize}
\item Vetor de interrupções para atender até 256 ISR \textit{(Interrupt Service Routines)}
\item E/S baseada em memória compartilhada, com controlador próprio, capaz de atender pelo menos 4 dispositivos de controle externos, com registradores de controle/dados mapeados em endereços do segmento de dados.
\item PIC \textit{(Programmable Interrupt Controller)} externo ao processador, com controle de mascaramento de interrupções por memória compartilhada, no segmento de dados.
\end{itemize}
\item Conjunto de intruções: tabela 1
\end{itemize}

\section{DESCRIÇÃO DO PROTÓTIPO}

\begin{figure}[thpb]
      \centering
      %\includegraphics[scale=1.0]{figurefile}
      %\framebox{\parbox{3in}{fig_1.png}}
      \caption{Arquitetura Pipeline}
      \label{}
   \end{figure}

\section{CONCLUÕES}

\subsection{Problemas Encontrados}

\subsection{Abrangência da Implementação}

\subsection{Possíveis Melhorias}

\subsection{Participação e aprendizado}

\addtolength{\textheight}{-12cm}   

\begin{thebibliography}{99}

\bibitem{c1} 
\bibitem{c2} 

\end{thebibliography}

\end{document}
